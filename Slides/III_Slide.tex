\documentclass[aspectratio=1610]{beamer}

\usetheme{metropolis}
\usepackage{appendixnumberbeamer}

\usepackage{booktabs}
\usepackage[scale=2]{ccicons}

\usepackage{pgfplots}
\usepgfplotslibrary{dateplot}

\usepackage{xspace}
\newcommand{\themename}{\textbf{\textsc{metropolis}}\xspace}

\usefonttheme{professionalfonts}   % required for mathspec
\usepackage{mathspec}
\usepackage{cmbright}
\setsansfont{Helvetica Neue}
\setmathsfont{Helvetica Neue}

\usecolortheme[snowy]{owl}
\setbeamercolor{section in toc}{fg=black}

\title{Limits and continuity}
\author{Yichao Huang, University of Helsinki}

\begin{document}

\maketitle

\begin{frame}{Week III: Outline}
\begin{enumerate}
  \item Limit of a function\newline
  \item Say hello to epsilon-delta\newline
  \item Continuous functions\newline
  \item Bolzano's theorem
\end{enumerate}
\end{frame}

\section{Symbols (and logic)}

\begin{frame}{Logic in every day life}
\end{frame}

\section{Examples of proofs}

\section{Set, elements, subsets}

\begin{frame}{What is a set?}
A \textbf{set} is a \textcolor{green}{well-defined} \textcolor{blue}{collection} of \textcolor{yellow}{distinct} \textcolor{violet}{objects}.
\end{frame}

\section{Counting finite sets}

\end{document}