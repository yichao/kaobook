\documentclass[aspectratio=1610]{beamer}

\usetheme{metropolis}
\usepackage{appendixnumberbeamer}

\usepackage{booktabs}
\usepackage[scale=2]{ccicons}

\usepackage{pgfplots}
\usepgfplotslibrary{dateplot}

\usepackage{xspace}
\newcommand{\themename}{\textbf{\textsc{metropolis}}\xspace}

\usefonttheme{professionalfonts}   % required for mathspec
\usepackage{mathspec}
\usepackage{cmbright}
\setsansfont{Helvetica Neue}
\setmathsfont{Helvetica Neue}

\usecolortheme[snowy]{owl}
\setbeamercolor{section in toc}{fg=black}

\title{Foundations of undergraduate analysis}
\author{Yichao Huang, University of Helsinki}

\begin{document}

\maketitle

\begin{frame}{Introduction to the course}
Course format:
\begin{itemize}
	\item Lecture on Monday/Tuesday: notes/slides available on the course web page.
	\item Exercice on Friday: exercices will be graded (30p).
	\item Quiz on Tuesday: about 20 minutes each (20p).
	\item Final exam (50p).
\end{itemize}

Advices and recommendations:
\begin{itemize}
  \item Give feedback and interact with others.
  \item Do exercices (several times), ask questions!
  \item Find your own rythme and form good habits. Think long-term!
\end{itemize}
\end{frame}

\begin{frame}{Zoom: pros and cons}
Bonus points:
\begin{itemize}
	\item You can drink and eat while attending the course!
	\item Arguably the future of learning method.
\end{itemize}

Minus points and how to fight against them:
\begin{itemize}
	\item Online material are available everywhere (e.g. MIT). Take advantage of the fact that your teacher is a real person and can adjust to your needs: give feedbacks and ask questions!
	\item Less social interaction and more individual works. Try to form study groups, organize regular meeting and try to see each other in real life.
	\item Learning environment can be too relaxed. Keep taking notes during the course: you memorize things better if you do it yourself!
\end{itemize}
\end{frame}

\begin{frame}{Week I: Outline}
\begin{enumerate}
  \item Mathematical symbols: $\forall, \exists, \Rightarrow$ etc.\newline
  \item Mathematical induction\newline
  \item Basic elements of set theory\newline
  \item A taste of combinatorics\newline
  \item Infimum and supremum\newline
  \item An epsilon of room
\end{enumerate}
\end{frame}

\section{Symbols (and logic)}

\begin{frame}{Logic in every day life}
\end{frame}

\section{Examples of proofs}

\section{Set, elements, subsets}

\begin{frame}[t]{What is a set?}
A \textbf{set} is a \textcolor{green}{well-defined} \textcolor{blue}{collection} of \textcolor{yellow}{distinct} \textcolor{violet}{objects}.
\end{frame}

\section{Counting finite sets}

\section{Infimum and supremum}

\section{Say hello to epsilon}

\end{document}